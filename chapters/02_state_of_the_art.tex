Přehled aktuálního stavu řešené problematiky podrobně shrnuje (1) současný stav poznání a výchozí podmínky pro řešení a to jak v tuzemsku, tak i v zahraničí (2) definuje problém, který je nutno a který se bude v práci řešit.

Tato část práce je převážně vytvořena jako rešerše za použití mnoha literárních zdrojů. 
Při výkladu se postupuje od obecnějších informací k informacím co nejkonkrétnějším a od toho, co se o dané problematice ví, k tomu, co je neznámé a aktuálně vhodné k řešení. Z~takto uspořádaného výkladu pak logicky vyplynou cíle práce vytyčené níže.